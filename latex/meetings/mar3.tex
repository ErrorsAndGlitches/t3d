\ssn{3 March 2013}

\sssn{Pre-meeting discussion topics/ideas}

Following the discussion of bonuses from the previous meeting, Benjamyn wanted
to discuss his idea of bonus upgrade. The following is a list of ideas for
combining bonuses. However, bonus combination could be implemented after the
game has been more fully developed. Bonuses will have a limit on their
upgradability. By upgrading a bonus, this also opens up a bonus slot.
\benum
	\bolditem{bomb}{the radius of the bomb increases. The un-upgraded bomb has a
		single block radius and the radius increases by one block from every upgrade,
		reaching a maximum radius of 3 blocks.}
	\bolditem{slow down/speed up}{changes the rate by an extra 0.5x than if the
		speed was changed using each of these bonuses individually. Thus upgrading a
		slow down, will slow down the speed by 2.5x rather than 2x.}
	\bolditem{chain}{the lifetime of the chain increases}
	\bolditem{block}{the lifetime of the block increases}
\eenum

Benjamyn also wanted to discuss whether the block bonus can be used to complete
a layer.

\para{Double reward}
Currently, if a player completes a level, they are rewarded twice: once by
completing a level and reducing the distance that the blocks are to their
ceiling, and once for pushing the platform towards their opponent's ceiling.
Thus, after completing a level, their top most block actually moves a distance
of two units away from the ceiling.

Benjamyn had the following idea: depending on how many layers the player
completes, various rewards are given:
\bitem
	\bolditem{one layer}{all that occurs is that the layer disappears and the
		above blocks fall as in the original tetris}
	\bolditem{two layers}{the layer is removed and they receive a bonus}
	\bolditem{three layers}{the probability that the bonus they receive is more
		valuable increases}
	\bolditem{four (max) layers}{the platform changes location towards their
		opponent's ceiling and away from their ceiling}
\eitem

\para{Coding style/layout}

Benjamyn wanted to discuss the coding style/layout.

\para{Calendar}

We decided on 26 February in class that the main goal of the next meeting was
to develop a timeline for the project and to continue the previous meeting's
discussions.

\sssn{In-meeting discussion}

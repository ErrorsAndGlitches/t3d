\ssn{8 February 2013}

\para{Game Introduction}

During this meeting, Benjamyn introduced to Zach his game idea, T3D, a 3D
tetris implementation that extends the gameplay of tetris that we all know and
love.  The game focuses on a battling ideology in which each player is trying
to push the platform away from their ceiling, causing the platform to get
closer to the opponent's ceiling. As in the original teris, if one of the
player's blocks falls above their ceiling, they loose. Figure \ref{fig:screen}
shows the main view that a player will see when playing the game. The
following are notes on the discussion in which Benjamyn and Zach fleshed out
their game. 

\figwpdf{meetings/pics}{screen}{A sketch of the screen that a player will see
on the main view of T3D.}{0.6}

\para{Bonuses}

The game also incorporates bonuses to extend the game play. A player will have
a limited number of slots to fill bonuses. The player can also choose which
subarena that they want to use the bonus on. Below is a list of bonuses and a
short description that we brainstormed during the meeting:
\benum
	\bolditem{bomb}{destroys a group of blocks in a radius of where the bomb
		detonates.}
	\bolditem{slow down/speed up}{changes the rate at which the blocks fall.}
	\bolditem{chain}{a chain across a subarena is produced which serves as an
		irritation for the player whose subarena the chain is applied to. This bonus
		expires after a certain elapsed time.}
	\bolditem{block}{a block is put into one of the subarenas. The block expires
		after a certain elapsed time.}
	\bolditem{change subarena height}{this bonus changes the height of the chosen
		subarena by a single layer. The bonus can be used in two ways: increasing or
		decreasing the height of the chosen subarena. This bonus expires after a
		certain elapsed time.}
	\bolditem{layer duplication/destruction}{duplicates or destroys a layer in
		the chosen subarena.}
\eenum

Bonuses were originally going to generated at random in holes of the subarena
or on the surface of the blocks in a subarena. A distribution on how powerful
each of the bonuses are to find a gameplay balance would need to be explored.
However, Zach came up with a better idea of generated the bonuses when a player
completes a layer (26 Feb 2013).

\para{Tokens}

After discussing some of the properties and gameplay of the game, we made a
list of tokens and subtokens that would be included in the game:
\bitem
	\item super block
		\sitem{block}{would contain state information: solid, bonus, gone}
	\item arena
	\bitem
		\item subarenas
			\sitem{layers}
		\item platform
		\item wall
	\eitem
	\item player
		\sitem{bonus set}
	\item game network and game network packets
\eitem

\para{Architecture}

Figure \ref{fig:arch} shows a diagram of the architecture that was producing
during our discussion.

\figwpdf{meetings/pics}{arch}{A diagram of the soft architecture of the T3D
game.}{0.6}

\para{Code properties}

We decided on using \href{https://bitbucket.org/}{bitbucket} for our code
repository because bitbucket allows for private repositories. We also decided
to use \href{http://www.stack.nl/~dimitri/doxygen/}{Doxygen} to automatically
produce documentation for our project.
